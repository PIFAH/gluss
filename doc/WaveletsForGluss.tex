\documentclass[11pt]{article}
\usepackage{geometry} % see geometry.pdf on how to lay out the page. There's lots.
\usepackage{hyperref}
\usepackage{graphicx}
\usepackage{gensymb}
\usepackage[affil-it]{authblk}
\usepackage[toc,page]{appendix}
\usepackage{pifont}
\usepackage{amsmath}
\usepackage{draftwatermark}

\SetWatermarkText{DRAFT}
\SetWatermarkScale{6}
\SetWatermarkLightness{0.95}

% \geometry{letter} % or letter or a5paper or ... etc
% \geometry{landscape} % rotated page geometry

% See the ``Article customise'' template for come common customisations

\title{Gluss = Slug + Truss}
\author{Robert L. Read
  \thanks{read.robert@gmail.com}
}
\affil{Founder, Public Invention, an educational non-profit.}


\date{\today}

%%% BEGIN DOCUMENT
\begin{document}

\maketitle

%% \tableofcontents

\section{Introduction}

This is a draft of an attempt to apply wavelet technology to gluss.
However, it may touch on pure and applied mathematics in different valuable ways.
\begin{itemize}  
\item We will argue that wavelet and Fourier transforms should be reformulated to
  accept the possibility a function being undefined at various points. Possibly we will
  learn later that this is wrong, but must provide a means of dealing with it.
\item We will argue that the tendency to treat solid objects as a curved surface in 3-space
  is insufficient for the specific purpose of mapping abstractly from objects to gluss gaits
  and for doing form-matching in three space.
\item We will argue that wavelets provide an ideal way to define gaits for robots,
  not as a transform, but as the original means for constructing and representing them.
\item We will argue that wavelets provide a means of matching solid shapes in both space and time.
\end{itemize}  

\section{Solid Shape Matching Via Wavelets}

A fundamental problem posed by an attempt to create a metamorphic material is how to specify and controls
its shape in both space and time. Ordinarily we would refer to the change in shape over time as a ``gait''
if its purpose is to locomote.

Gluss consists of a number of actuators connected by joints.  Likewise tensegrity robots might consist
of very numerous colletions of struts connected via cables. Whether we consider a moving robot or
a static structure, we would like to conveniently compute a configure of a such machines to match
any solid shape.

It is typical to operate in terms of surfaces. However, we argue it is far better to use three-dimensional
objects directly. It is entirely reasonable to create a general shape in 3-space as a collection of
objects which are not in fact in any way connected, and yet still compute a matching shape of a
connected gluss.  Or, if dealing with a robot swarm, we might even relax the constraint that the result
be connected.

Our fundamental problem is to compute a mapping from what we might call the ``Blob Space'', $\mathcal{B}$, into
the ``Gluss Space'', $\matchal{G}$. The properties of the the Blob Space are:
\begin{itemize}  
\item It should be really easy for human beings to design objects in $\mathcal{B}$.
\item It should be easy to import representations of objects into $\mathcal{B}$.
\item It should be easy to animate objects in $\mathcal{B}$ without constraints imposed by
  the physical world.  For example, motion need not be continuous and mass need not be conserved.
\item It is untainted by any notion of scale.
\end{itemize}  

The Gluss Space, on the other hand:
\begin{itemize}  
\item Imposes strict limitations on geometry to match the physical limitations of actual robots.
\item An object in $\mathcal{G}$ can be directly implemented with a simulation or a physical robot.
\item Furthermore time may have limits as well. Motion in $\mathcal{G} must be continuous, and
  might have to be slow.
\end{itemize}  

Let us consider an example.  We wish to explore the possibility of making a gluss-style robot
walk to  building and crawl through a window. This is requires locomotion, elevation, and
potentially decreasing the size of the robot to fit the window.  Even if the glussbot can
jump through the window, it must potentially deal with these issues, and we may plausible
assume that it will crawl through the window instead, like a worm.

It would be convenient to model the gluss bot as 6 or 7 spheres in \mathcal{B} of the same size.
These need not really be connected, they should just be close enough that they generally form
a stout worm, or grub-like, shape.  Then we develop a short series of steps on \matchal{B}, not
even a smooth animation, that suggests getting through the window. Let us imagine 10 such steps.
This could probably be done in 3 minutes using a mouse and some javascript code.

Then we want to specify a Gluss Space matching a glussbot we actually have. We would like
to automatically map the Blob Space steps into a Gluss Space motion that potentially gets our
glussbot through the window.

It is possible that work done in animation will assist us greatly in this already; I need
to research that.

Assuming this is not already a solved problem, I claim that 3D wavelets are the best way
to approach the mapping problem.

A second example.  In order to give wonder to a child, we want to order our glussbot to
assume the shape of a lion. We import an .stl file into the Blob Space.  But we have a highly
constrained gluss bot.  We have to figure out how to order it to approximate the lion shape.
This problem could be approached as a problem of conformity to an outside surface, but I think
it will be easier when approached as a problem of matching masses described as 3D wavelets.



\section{}


\end{document}
